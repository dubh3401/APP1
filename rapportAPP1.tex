% ======================================================================


% Sujet du document
% Informations importantes
%
%
% Prénom Nom
% H. Dube 2019
% ======================================================================
% Ce code rassemble les efforts d'étudiants de la faculté de génie  
% de l'université de Sherbrooke afin de faire un template LaTeX moderne
% dédié à l'écriture de rapport universitaire.
% Ce document est libre d'être utilisé et modifié.
% ======================================================================

% ----------------------------------------------------
% Initialisation
% ----------------------------------------------------
\documentclass{udes_rapport} % Voir udes_rapport.cls

\begin{document}
\selectlanguage{french}

% ----------------------------------------------------
% Configurer la page titre
% ----------------------------------------------------

% Information
\faculte{Génie}
\departement{génie électrique et génie informatique}
\app{1}{Éléments de statique et de dynamique}
\professeur{M. Ahmed Khoumsi et M. Raef Cherif}
\etudiants{Hubert Dubé - dubh3401 \\ Marc Sirois - sirm2508\\ Gabriel Lavoie - lavg2007}
\dateRemise{4 septembre 2019}


% ======================================================================
\pagenumbering{roman} % met les numéros de pages en romain
% ----------------------------------------------------
% Page titre
% ----------------------------------------------------
\fairePageTitre{LOGO} % Options: [STD, LOGO]
\newpage

% ----------------------------------------------------
% Table des matières
% ----------------------------------------------------
\tableofcontents
\newpage


% ----------------------------------------------------
% Table des figures
% ----------------------------------------------------
\listoffigures
\newpage



% ======================================================================
% Document
% ======================================================================
\pagenumbering{arabic} % met des chiffres arabes
\setcounter{page}{1} % reset les numéros de pages
%%%%%%%%%%%%%%%%%%%%%%%%%%%%%%%%%%%%%%%%%%%%%%%%%%%%%%
%{Analyse du signal
%%%%%%%%%%%%%%%%%%%%%%%%%%%%%%%%%%%%%%%%%%%%%%%%%%%%%%
\section{Introduction}

L'objectif de ce rapport est d'étudier la mécanique d'un robot manipulateur. Premièrement, la cinématique de l'extrémité mobile est analysée dans le cas des mouvements généraux, ainsi que dans les cas particuliers de mouvements horizontaux et verticaux. Deuxièmement, une étude approfondie de la statique et la dynamique du robot est effectuée, afin d'assurer son équilibre et d'aider au choix des moteurs.

\section{Cinématique}
\subsection{Mouvement de A dans le cas général}
Le positionnement de $\overrightarrow{OA}$ peut être exprimé par l'addition:
\begin{equation}
	\overrightarrow{OA} = \overrightarrow{OB} + \overrightarrow{BA}
\end{equation}
	\[	\overrightarrow{OA_x} = l_1 cos(\theta) + l_2 cos(\phi) 			\]
	\[	\overrightarrow{OA_y} = l_1 sin(\theta) + l_2 sin(\phi)			\]


la vitesse étant la dérivée de la position :
\begin{equation}
\overrightarrow{V_A} = \frac{d\overrightarrow{OA}}{dt}
\end{equation}

\[ \overrightarrow{V_Ax} = \frac{d(\overrightarrow{OA_x}}{dt} = \frac{d(l_1 cos(\theta) + l_2 cos(\phi))}{dt}	\]
\[ \overrightarrow{V_Ax} = -l_1 sin(\theta) \dot{\theta} -l_2 sin(\phi) \dot{\phi}								\]
\[ \overrightarrow{V_Ay} = \frac{d(\overrightarrow{OA_y}}{dt} = \frac{d(l_1 sin(\theta) + l_2 sin(\phi))}{dt}	\]
\[ \overrightarrow{V_Ax} = l_1 cos(\theta) \dot{\theta} -l_2 cos(\phi) \dot{\phi}								\]


La même stratégie peut être utilisé pour obtenir l'accélération :
\begin{equation}
\overrightarrow{a_A} = \frac{d\overrightarrow{V_A}}{dt}	\\
\end{equation}
\[	\overrightarrow{a_Ax} = \frac{d\overrightarrow{OA_x}}{dt} = \frac{ d(l_1 cos(\theta) + l_2 cos(\phi))}{dt}		\]
\[	\overrightarrow{a_Ay} = \frac{d\overrightarrow{OA_y}}{dt} = \frac{d(l_1 cos(\theta) + l_2 cos(\phi))}{dt}		\]

\subsection{Mouvement horizontal de A}

\noindent\begin{minipage}{\textwidth} 
\begin{minipage}{0.5\textwidth}
  \centering
  \includegraphics[width=.75\linewidth]{pos_hori_0}
  \captionof{subfigure}{Position initiale}
  \label{pos_hori:position_horizontal_initiale}
\end{minipage}%
\begin{minipage}{0.5\textwidth}
  \centering 
  \includegraphics[width=.75\linewidth]{pos_hori_pi3} 
  \captionof{subfigure}{Position finale} 
  \label{pos_hori:position_horizontal_finale} 
\end{minipage} 
\captionof{figure}{Position du mouvement horizontale} 
\label{pos_hori} 
\end{minipage}

En position initiale, la distance entre le moteur O et le poids est de 2L. 
En position finale, la distance OA forme un triangle équilatéral avec les deux bras, puisque les trois
angles sont de pi/3

\begin{center}
	\centering
	\includegraphics[width=0.7\textwidth]{theta_hori}
	\captionof{figure}{Composantes en fonction de $\protect\theta$}
	\label{composantes_horizontale_theta}
\end{center}

On observe que la position débute à 0,5 mètres, soit la longueur des deux bras, et termine à 0,25 mètres, soit la distance l.
La vitesse est nulle au départ et diminue alors que l'angle augmente. La vitesse angulaire est de $25\:rad/s$, dans le sens antihoraire, ce qui est
traduit en vitesse linéaire négative pour le point A. Cette vitesse explique la grande valeur négative de l'accélération linéaire horizontale.

\subsection{Mouvement vertical de A}

\noindent\begin{minipage}{\textwidth} 
\begin{minipage}{0.5\textwidth}
  \centering
  \includegraphics[width=.75\linewidth]{pos_vert_0}
  \captionof{subfigure}{Position initiale}
  \label{pos_vert:position_verticale_initiale}
\end{minipage}%
\begin{minipage}{0.5\textwidth}
  \centering 
  \includegraphics[width=.75\linewidth]{pos_vert_pi3} 
  \captionof{subfigure}{Position finale} 
  \label{pos_vert:position_verticale_finale} 
\end{minipage} 
\captionof{figure}{Position du mouvement vertical} 
\label{pos_vert} 
\end{minipage}
\\ \\
Pour la position initiale dans le cas du mouvement vertical, on pose un angle $\varphi$ initial de $\pi/2$. 
Pour conserver la distance horizontale de l, l'angle $\varphi$ diminue lorsque $\theta$ augmente. 
Comme $\theta$ a une valeur de $\pi/3$ en position finale, on n'atteint jamais la situation ou $\overrightarrow{OA} = 2l$, 
ce qui impliquerais un angle $\theta$ de $\pi/4$.

\begin{center}
	\centering
	\includegraphics[width=0.7\textwidth]{theta_vert}
	\captionof{figure}{Composantes en fonction de $\protect\theta$}
	\label{composantes_verticale_theta}
\end{center}

On remarque premièrement que la position verticale augmente avec la progression de l'angle $\theta$, contrairement au mouvement horizontal.
D'ailleurs, la vitesse est à son maximum au début de mouvement, ce qui s'explique par le fait qu'à $\theta=0$, 
la vitesse linéaire de A est entièrement tangentielle à la rotation de B.

\subsection{Analyse avec Matlab}


\section{Statique et dynamique}
\subsection{Statique}
La figure ci-dessous démontre les forces en action qui influencent le calcul de la force $F_b$ lorsque le robot est immobile.
\begin{center}
	\centering
	\includegraphics[width=0.3\textwidth]{statique_fb}
	\captionof{figure}{Diagramme des forces en statique pour le calcul de $F_b$}
	\label{statique_fb}
\end{center}
La somme des forces d'un système statique est égale à 0, tel que décrit par:
\begin{equation}
	\sum \overrightarrow F = \overrightarrow 0
\end{equation}
En suivant la formule et faisant la sommation des vecteurs de force, on obtient:
	\[	\sum \overrightarrow F = \overrightarrow 0 = \overrightarrow{F_b} + \overrightarrow{F_{BA}} + \overrightarrow{F_A}	\]
	\[	\begin{bmatrix}
	0\\ 
	0
	\end{bmatrix} = \begin{bmatrix}
	F_{bx}\\ 
	F_{by}
	\end{bmatrix} + \begin{bmatrix}
	0\\ 
	-F_{BA}
	\end{bmatrix} + \begin{bmatrix}
	0\\ 
	-F_A
	\end{bmatrix}	\]
On observe rapidement que le valeur de $F_{bx}$ est égale à 0 et que la valeur de $F_{by}$ correspond:
	\[	F_{by} = F_{BA} + F_A 	\]
En utilisant l'equation:
\begin{equation}
	F = m*g
\end{equation}
On obtient:
	\[	F_{by} = g*(m_{BA} + m_A)	\]
Et donc la valeur de Fb en statique, avec g étant l'accélération gravitationnelle:
	\[	F_b = \begin{bmatrix}
	0\\ 
	g*(m_{BA} + m_A)
	\end{bmatrix}	\]
La figure ci-dessous démontre les forces en action qui influencent le calcul du moment $C_b$:
\begin{center}
	\centering
	\includegraphics[width=0.3\textwidth]{statique_cb}
	\captionof{figure}{Diagramme des forces en statique pour le calcul de $C_b$}
	\label{statique_cb}
\end{center}
La somme des moments à un point (dans notre cas B) d'un système statique est égale à 0, tel que décrit par:
\begin{equation}
	\sum M_B =  0
\end{equation}
Les forces $\overrightarrow F_{BA}$ et $\overrightarrow F_A$ ont une influence tangentielle et normale à la tige BA. Pour le calcul des moments, uniquement la partie tangentielle nous intéresse (la normale n'a pas d'impact). La tangentielle se trouve à être la projection des forces ($cos\varphi$). On obtient alors:
	\[	\sum M_B =  0 = C_b - cos\varphi*F_{BA}*l_2/2 - cos\varphi*F_A*l_2	\]
En isolant $C_b$ et simplifiant l'équation avec les valeurs pour $F_{BA}$ et $F_A$ trouvées ci-haut, on obtient sa valeur pour le cas statique (avec g étant l'accélération gravitationnelle):
	\[	C_b = l_2*g*cos\varphi*(m_{BA}/2+m_A)	\]
\begin{center}
	\centering
	\includegraphics[width=0.7\textwidth]{couple_stat}
	\captionof{figure}{couple statique en fonction de $\protect\theta$}
	\label{couple_statique}
\end{center}
\subsection{Dynamique}
La figure ci-dessous démontre les forces en action qui influencent le calcul de la force $F_b$ lorsque le robot est immobile à l'exception de la tige BA qui a une accélération angulaire constante de $\alpha$.
\begin{center}
	\centering
	\includegraphics[width=0.3\textwidth]{dynamique_fb}
	\captionof{figure}{Diagramme des forces en dynamique pour le calcul de $F_b$}
	\label{dynamique_fb}
\end{center}
La somme des forces d'un système dynamique est égale à les accélérations fois les masses accélérées, tel que décrit par:
\begin{equation}
	\sum \overrightarrow F_{ext} = m*\overrightarrow {\gamma_G}
\end{equation}
L'accélération angulaire $\alpha$ cause des accélérations normale et tangentielle $\overrightarrow a_n$ et $\overrightarrow a_t$. Ces accélérations correspondant à:
\begin{equation}
	 a_n = l_2 * \dot{\varphi}^{2}
\end{equation}
\begin{equation}
	 a_t = l_2 * \ddot{\varphi}
\end{equation}
Cependant, afin de tout avoir en termes de l'axe des x et des y, on veut décomposer ces accélérations selon leurs projections sur les deux axes, tout en prenant compte que c'est tournant:
	\[	\overrightarrow {\gamma_G} = \begin{bmatrix}
	-sin\varphi*a_t-cos\varphi*a_n\\
	cos\varphi*a_t-sin\varphi*a_n
	\end{bmatrix}	\]
Et en séparant pour la barre BA et pour la masse A, en utilisant le équations pour les accélérations et en appliquant les masses:
	\[	m * \overrightarrow {\gamma_G} = (l_2*m_A + l_2/2*m_{BA}) * \begin{bmatrix}
	-\ddot{\varphi}*sin\varphi-\dot{\varphi}^{2}*cos\varphi\\
	\ddot{\varphi}*cos\varphi-\dot{\varphi}^{2}*sin\varphi
	\end{bmatrix}	\]
En suivant l'équation et en faisant la sommation des vecteurs de force, on obtient:
	\[	\sum \overrightarrow F_{ext} = m * \overrightarrow {\gamma_G} =  \overrightarrow{F_b} + \overrightarrow{F_{BA}} + \overrightarrow{F_A}	\]
	\[	(l_2*m_A + l_2/2*m_{BA}) * \begin{bmatrix}
	-\ddot{\varphi}*sin\varphi-\dot{\varphi}^{2}*cos\varphi\\
	\ddot{\varphi}*cos\varphi-\dot{\varphi}^{2}*sin\varphi
	\end{bmatrix} = \begin{bmatrix}
	F_{bx}\\ 
	F_{by}
	\end{bmatrix} + \begin{bmatrix}
	0\\ 
	-F_{BA}
	\end{bmatrix} + \begin{bmatrix}
	0\\ 
	-F_A
	\end{bmatrix}	\]
En utilisant l'equation:
\begin{equation}
	F = m*g
\end{equation}
On obtient pour le cas dynamique, avec g étant l'accélération gravitationelle:
	\[	\overrightarrow {F_b} =  \begin{bmatrix}
	F_{bx}\\ 
	F_{by}
	\end{bmatrix} = (l_2*m_A + l_2/2*m_{BA}) * \begin{bmatrix}
	-\ddot{\varphi}*sin\varphi-\dot{\varphi}^{2}*cos\varphi\\
	\ddot{\varphi}*cos\varphi-\dot{\varphi}^{2}*sin\varphi + g*(m_{BA}+m_A)
	\end{bmatrix}	\]
La figure ci-dessous démontre les forces en action qui influencent le calcul du couple $C_b$ lorsque le robot est immobile à l'exception de la tige BA qui a une accélération angulaire constante de $\alpha$.
\begin{center}
	\centering
	\includegraphics[width=0.3\textwidth]{dynamique_cb}
	\captionof{figure}{Diagramme des forces en dynamique pour le calcul de $C_b$}
	\label{dynamique_cb}
\end{center}
La somme des moments à un point (dans notre cas B) d'un système dynamique est égale au moment d'inertie multiplié par l'accélération angulaire, tel que décrit par:
\begin{equation}
	\sum M_B =  I_B*\alpha
\end{equation}
Le moment d'intertie au point B est composé en trois parties, le moment d'inertie du moteur $M_b$, le moment d'inertie de la tige BA, et le moment d'inertie de l'objet $O_A$, tous calculées avec le centre de rotation B:
	\[	I_B = I_{M_B} + I_BA + I_A 	\]
 Le moment d'inertie des sphères $M_B$ et $O_A$ est négligible par rapport à la tige et la masse, alors le moment d'inertie au point B est:
	\[	I_B = 0 + m_{BA}/3*{l_2}^{2} + 0 + m_A*{l_2}^{2} =  (m_{BA}/3 + m_A)*{l_2}^{2}	\]
En utilisant cette information et en procédant à la sommation des moments:
	\[	\sum M_B =  I_B*\alpha =  (m_{BA}/3 + m_A)*{l_2}^{2}*\alpha = C_b - cos\varphi*F_{BA}*l_2/2 - cos\varphi*F_A*l_2   \]
On obtient alors, avec g étant l'accélération gravitationnelle et $\ddot{\varphi} = \alpha$:
	\[	C_b = (m_{BA}/3 + m_A)*{l_2}^{2}*\ddot{\varphi} +  l_2*g*cos\varphi*(m_{BA}/2+m_A)	\]
\begin{center}
	\centering
	\includegraphics[width=0.7\textwidth]{couple_dyn}
	\captionof{figure}{couple dynamique en fonction de $\protect\theta$}
	\label{couple_dynamique}
\end{center}
\subsection{Analyse avec Matlab}

\section{Conclusion}

L'étude des forces et des couples en dynamique et statique saura aider à choisir les moteurs idéaux pour la réalisation du projet, et les équation de cinématique seront essentielles pour en programmer les mouvements. En bref, l'analyse développée dans ce rapport contribuera au développement du robot. Cependant, il sera essentiel d'approfondir l'étude en incluant les mouvements de rotation du bras $\overrightarrow{XO}$ afin d'obtenir une modélisation complète de sa mécanique. Cette étude sera réalisée ultérieurement.

\begin{comment}
\begin{center}
	\centering
	\includegraphics[width=0.7\textwidth]{puissance}
	\captionof{figure}{Spectre de puissance d'une onde de 1kHz}
	\label{puissance}
\end{center}


\section{Filtres FIR}
\noindent\begin{minipage}{\textwidth} 
\begin{minipage}{0.5\textwidth}
  \centering
  \includegraphics[width=.75\linewidth]{ampFIR}
  \captionof{subfigure}{Amplitude}
  \label{FIR:ampFIR}
\end{minipage}%
\begin{minipage}{0.5\textwidth}
  \centering 
  \includegraphics[width=.75\linewidth]{phaseCute} 
  \captionof{subfigure}{Phase} 
  \label{FIR:phaseFIR} 
\end{minipage} 
\captionof{figure}{Filtre IIR} 
\label{FIR} 
\end{minipage}
\end{comment}

\end{document}












